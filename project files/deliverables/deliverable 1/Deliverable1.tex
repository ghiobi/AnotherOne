
%----------------------------------------------------------------------------------------
%	PACKAGES AND OTHER DOCUMENT CONFIGURATIONS
%----------------------------------------------------------------------------------------

\documentclass[12pt]{article}
\usepackage[english]{babel}
\usepackage[utf8x]{inputenc}
\usepackage{amsmath}
\usepackage{graphicx}
\usepackage{afterpage}
\usepackage{todonotes}
\usepackage[margin=1.2in]{geometry}
\usepackage{listings}

% Mark's packages...trust in MOW
\usepackage{float}		% this is to place figures where requested!
\usepackage{times}		% this uses fonts which will look nice in PDF format
\usepackage{graphicx}		% needed for the figures
\usepackage{url}

%This does NOT specify language
\definecolor{mygreen}{rgb}{0,0.4,0}
\definecolor{mypurp}{rgb}{0.5,0,0.5}

\lstdefinelanguage{JavaScript}{
  keywords={typeof, new, true, false, catch, function, return, null, catch, switch, var, if, in, while, do, else, case, break},
  ndkeywords={class, export, boolean, throw, implements, import, this},
  ndkeywordstyle=\color{darkgray}\bfseries,
  identifierstyle=\color{black},
  sensitive=false,
  comment=[l]{//},
  morecomment=[s]{/*}{*/},
  morestring=[b]',
  morestring=[b]"
}

 \lstdefinelanguage{CSS}{
      keywords={accelerator,azimuth,background,background-attachment,
            background-color,background-image,background-position,
            background-position-x,background-position-y,background-repeat,
            behavior,border,border-bottom,border-bottom-color,
            border-bottom-style,border-bottom-width,border-collapse,
            border-color,border-left,border-left-color,border-left-style,
            border-left-width,border-right,border-right-color,
            border-right-style,border-right-width,border-spacing,
            border-style,border-top,border-top-color,border-top-style,
            border-top-width,border-width,bottom,caption-side,clear,
            clip,color,content,counter-increment,counter-reset,cue,
            cue-after,cue-before,cursor,direction,display,elevation,
            empty-cells,filter,float,font,font-family,font-size,
            font-size-adjust,font-stretch,font-style,font-variant,
            font-weight,height,ime-mode,include-source,
            layer-background-color,layer-background-image,layout-flow,
            layout-grid,layout-grid-char,layout-grid-char-spacing,
            layout-grid-line,layout-grid-mode,layout-grid-type,left,
            letter-spacing,line-break,line-height,list-style,
            list-style-image,list-style-position,list-style-type,margin,
            margin-bottom,margin-left,margin-right,margin-top,
            marker-offset,marks,max-height,max-width,min-height,
            min-width,-moz-binding,-moz-border-radius,
            -moz-border-radius-topleft,-moz-border-radius-topright,
            -moz-border-radius-bottomright,-moz-border-radius-bottomleft,
            -moz-border-top-colors,-moz-border-right-colors,
            -moz-border-bottom-colors,-moz-border-left-colors,-moz-opacity,
            -moz-outline,-moz-outline-color,-moz-outline-style,
            -moz-outline-width,-moz-user-focus,-moz-user-input,
            -moz-user-modify,-moz-user-select,orphans,outline,
            outline-color,outline-style,outline-width,overflow,
            overflow-X,overflow-Y,padding,padding-bottom,padding-left,
            padding-right,padding-top,page,page-break-after,
            page-break-before,page-break-inside,pause,pause-after,
            pause-before,pitch,pitch-range,play-during,position,quotes,
            -replace,richness,right,ruby-align,ruby-overhang,
            ruby-position,-set-link-source,size,speak,speak-header,
            speak-numeral,speak-punctuation,speech-rate,stress,
            scrollbar-arrow-color,scrollbar-base-color,
            scrollbar-dark-shadow-color,scrollbar-face-color,
            scrollbar-highlight-color,scrollbar-shadow-color,
            scrollbar-3d-light-color,scrollbar-track-color,table-layout,
            text-align,text-align-last,text-decoration,text-indent,
            text-justify,text-overflow,text-shadow,text-transform,
            text-autospace,text-kashida-space,text-underline-position,top,
            unicode-bidi,-use-link-source,vertical-align,visibility,
            voice-family,volume,white-space,widows,width,word-break,
            word-spacing,word-wrap,writing-mode,z-index,zoom},  
      sensitive=true,
      morecomment=[l]{//},
      morecomment=[s]{/*}{*/},
      morestring=[b]',
      morestring=[b]",
      alsoletter={:},
      alsodigit={-}
    }
    \lstdefinelanguage{HTML5}{
            language=html,
            sensitive=true, 
            alsoletter={<>=-},
            otherkeywords={
            % HTML tags
            <, </, >,
            </a, <a, </a>,
            </abbr, <abbr, </abbr>,
            </address, <address, </address>,
            </area, <area, </area>,
            </area, <area, </area>,
            </article, <article, </article>,
            </aside, <aside, </aside>,
            </audio, <audio, </audio>,
            </audio, <audio, </audio>,
            </b, <b, </b>,
            </base, <base, </base>,
            </bdi, <bdi, </bdi>,
            </bdo, <bdo, </bdo>,
            </blockquote, <blockquote, </blockquote>,
            </body, <body, </body>,
            </br, <br, </br>,
            </button, <button, </button>,
            </canvas, <canvas, </canvas>,
            </caption, <caption, </caption>,
            </cite, <cite, </cite>,
            </code, <code, </code>,
            </col, <col, </col>,
            </colgroup, <colgroup, </colgroup>,
            </data, <data, </data>,
            </datalist, <datalist, </datalist>,
            </dd, <dd, </dd>,
            </del, <del, </del>,
            </details, <details, </details>,
            </dfn, <dfn, </dfn>,
            </div, <div, </div>,
            </dl, <dl, </dl>,
            </dt, <dt, </dt>,
            </em, <em, </em>,
            </embed, <embed, </embed>,
            </fieldset, <fieldset, </fieldset>,
            </figcaption, <figcaption, </figcaption>,
            </figure, <figure, </figure>,
            </footer, <footer, </footer>,
            </form, <form, </form>,
            </h1, <h1, </h1>,
            </h2, <h2, </h2>,
            </h3, <h3, </h3>,
            </h4, <h4, </h4>,
            </h5, <h5, </h5>,
            </h6, <h6, </h6>,
            </head, <head, </head>,
            </header, <header, </header>,
            </hr, <hr, </hr>,
            </html, <html, </html>,
            </i, <i, </i>,
            </iframe, <iframe, </iframe>,
            </img, <img, </img>,
            </input, <input, </input>,
            </ins, <ins, </ins>,
            </kbd, <kbd, </kbd>,
            </keygen, <keygen, </keygen>,
            </label, <label, </label>,
            </legend, <legend, </legend>,
            </li, <li, </li>,
            </link, <link, </link>,
            </main, <main, </main>,
            </map, <map, </map>,
            </mark, <mark, </mark>,
            </math, <math, </math>,
            </menu, <menu, </menu>,
            </menuitem, <menuitem, </menuitem>,
            </meta, <meta, </meta>,
            </meter, <meter, </meter>,
            </nav, <nav, </nav>,
            </noscript, <noscript, </noscript>,
            </object, <object, </object>,
            </ol, <ol, </ol>,
            </optgroup, <optgroup, </optgroup>,
            </option, <option, </option>,
            </output, <output, </output>,
            </p, <p, </p>,
            </param, <param, </param>,
            </pre, <pre, </pre>,
            </progress, <progress, </progress>,
            </q, <q, </q>,
            </rp, <rp, </rp>,
            </rt, <rt, </rt>,
            </ruby, <ruby, </ruby>,
            </s, <s, </s>,
            </samp, <samp, </samp>,
            </script, <script, </script>,
            </section, <section, </section>,
            </select, <select, </select>,
            </small, <small, </small>,
            </source, <source, </source>,
            </span, <span, </span>,
            </strong, <strong, </strong>,
            </style, <style, </style>,
            </summary, <summary, </summary>,
            </sup, <sup, </sup>,
            </svg, <svg, </svg>,
            </table, <table, </table>,
            </tbody, <tbody, </tbody>,
            </td, <td, </td>,
            </template, <template, </template>,
            </textarea, <textarea, </textarea>,
            </tfoot, <tfoot, </tfoot>,
            </th, <th, </th>,
            </thead, <thead, </thead>,
            </time, <time, </time>,
            </title, <title, </title>,
            </tr, <tr, </tr>,
            </track, <track, </track>,
            </u, <u, </u>,
            </ul, <ul, </ul>,
            </var, <var, </var>,
            </video, <video, </video>,
            </wbr, <wbr, </wbr>,
            />, <!
            },  
            ndkeywords={
            % General
            =,
            % HTML attributes
            accept=, accept-charset=, accesskey=, action=, align=, alt=, async=, autocomplete=, autofocus=, autoplay=, autosave=, bgcolor=, border=, buffered=, challenge=, charset=, checked=, cite=, class=, code=, codebase=, color=, cols=, colspan=, content=, contenteditable=, contextmenu=, controls=, coords=, data=, datetime=, default=, defer=, dir=, dirname=, disabled=, download=, draggable=, dropzone=, enctype=, for=, form=, formaction=, headers=, height=, hidden=, high=, href=, hreflang=, http-equiv=, icon=, id=, ismap=, itemprop=, keytype=, kind=, label=, lang=, language=, list=, loop=, low=, manifest=, max=, maxlength=, media=, method=, min=, multiple=, name=, novalidate=, open=, optimum=, pattern=, ping=, placeholder=, poster=, preload=, pubdate=, radiogroup=, readonly=, rel=, required=, reversed=, rows=, rowspan=, sandbox=, scope=, scoped=, seamless=, selected=, shape=, size=, sizes=, span=, spellcheck=, src=, srcdoc=, srclang=, start=, step=, style=, summary=, tabindex=, target=, title=, type=, usemap=, value=, width=, wrap=,
            % CSS properties
            accelerator:,azimuth:,background:,background-attachment:,
            background-color:,background-image:,background-position:,
            background-position-x:,background-position-y:,background-repeat:,
            behavior:,border:,border-bottom:,border-bottom-color:,
            border-bottom-style:,border-bottom-width:,border-collapse:,
            border-color:,border-left:,border-left-color:,border-left-style:,
            border-left-width:,border-right:,border-right-color:,
            border-right-style:,border-right-width:,border-spacing:,
            border-style:,border-top:,border-top-color:,border-top-style:,
            border-top-width:,border-width:,bottom:,caption-side:,clear:,
            clip:,color:,content:,counter-increment:,counter-reset:,cue:,
            cue-after:,cue-before:,cursor:,direction:,display:,elevation:,
            empty-cells:,filter:,float:,font:,font-family:,font-size:,
            font-size-adjust:,font-stretch:,font-style:,font-variant:,
            font-weight:,height:,ime-mode:,include-source:,
            layer-background-color:,layer-background-image:,layout-flow:,
            layout-grid:,layout-grid-char:,layout-grid-char-spacing:,
            layout-grid-line:,layout-grid-mode:,layout-grid-type:,left:,
            letter-spacing:,line-break:,line-height:,list-style:,
            list-style-image:,list-style-position:,list-style-type:,margin:,
            margin-bottom:,margin-left:,margin-right:,margin-top:,
            marker-offset:,marks:,max-height:,max-width:,min-height:,
            min-width:,transition-duration:,transition-property:,
            transition-timing-function:,transform:,
            -moz-transform:,-moz-binding:,-moz-border-radius:,
            -moz-border-radius-topleft:,-moz-border-radius-topright:,
            -moz-border-radius-bottomright:,-moz-border-radius-bottomleft:,
            -moz-border-top-colors:,-moz-border-right-colors:,
            -moz-border-bottom-colors:,-moz-border-left-colors:,-moz-opacity:,
            -moz-outline:,-moz-outline-color:,-moz-outline-style:,
            -moz-outline-width:,-moz-user-focus:,-moz-user-input:,
            -moz-user-modify:,-moz-user-select:,orphans:,outline:,
            outline-color:,outline-style:,outline-width:,overflow:,
            overflow-X:,overflow-Y:,padding:,padding-bottom:,padding-left:,
            padding-right:,padding-top:,page:,page-break-after:,
            page-break-before:,page-break-inside:,pause:,pause-after:,
            pause-before:,pitch:,pitch-range:,play-during:,position:,quotes:,
            -replace:,richness:,right:,ruby-align:,ruby-overhang:,
            ruby-position:,-set-link-source:,size:,speak:,speak-header:,
            speak-numeral:,speak-punctuation:,speech-rate:,stress:,
            scrollbar-arrow-color:,scrollbar-base-color:,
            scrollbar-dark-shadow-color:,scrollbar-face-color:,
            scrollbar-highlight-color:,scrollbar-shadow-color:,
            scrollbar-3d-light-color:,scrollbar-track-color:,table-layout:,
            text-align:,text-align-last:,text-decoration:,text-indent:,
            text-justify:,text-overflow:,text-shadow:,text-transform:,
            text-autospace:,text-kashida-space:,text-underline-position:,top:,
            unicode-bidi:,-use-link-source:,vertical-align:,visibility:,
            voice-family:,volume:,white-space:,widows:,width:,word-break:,
            word-spacing:,word-wrap:,writing-mode:,z-index:,zoom:
            },  
            morecomment=[s]{<!--}{-->},
            tag=[s]
    }

\lstset{
    belowcaptionskip=1\baselineskip,
    breaklines=true,
    frame=single,
    xleftmargin=\parindent,
    showstringspaces=false,
    basicstyle=\scriptsize\ttfamily,
    keywordstyle=\color{blue},
    commentstyle=\color{mygreen},
    stringstyle=\color{mypurp},
    %
    %
    otherkeywords={...},           % add keywords
    deletekeywords={...},            % delete keywords
%
    keepspaces=true,                 % keeps spaces, usefull for indentation using spaces (might need columns=flexible)    breaklines=true,
    caption=\lstname,                % Makes the caption the filename
    breakautoindent=true,   % get the red arrows
    postbreak=\raisebox{0ex}[0ex][0ex]{\ensuremath{\color{red}\hookrightarrow\space}}
}



\begin{document}

\begin{titlepage}

\newcommand{\HRule}{\rule{\linewidth}{0.5mm}} % Defines a new command for the horizontal lines, change thickness here

\center % Center everything on the page
 
%----------------------------------------------------------------------------------------
%	HEADING SECTIONS
%----------------------------------------------------------------------------------------

\textsc{\LARGE Concordia University}\\[1cm] % Name of your university/college
\textsc{\Large SOEN 341}\\[1cm] % Major heading such as course name
\begin{flushleft} 
\end{flushleft}
\begin{minipage}{0.45\textwidth}
\begin{flushleft} \large
\textsc{\large Team Another One }
\end{flushleft}
\end{minipage}
\begin{minipage}{0.45\textwidth}
\begin{flushright} \large
\textsc{\large Mytinerary}
\end{flushright}
\end{minipage}\\[0.5cm]

%----------------------------------------------------------------------------------------
%	TITLE SECTION
%----------------------------------------------------------------------------------------

\HRule \\[0.4cm]
{ \huge \bfseries Deliverable 1}\\[0.4cm] % Title of your document
\HRule \\[1cm]
 
%----------------------------------------------------------------------------------------
%	AUTHOR SECTION
%----------------------------------------------------------------------------------------
\centering {\large{\emph{Group Members}}}\\
[0.5cm]
\begin{minipage}{0.45\textwidth}
\begin{flushleft} \large
Piratheeban \textsc{Annamalai}  \\
Laurendy \textsc{Lam}  \\
Jacqueline \textsc{Luo}  \\
Michael \textsc{Mescheder}  \\
Andy \textsc{Nguyen}    \\
Kenny \textsc{Nguyen}  \\
\end{flushleft}
\end{minipage}
\begin{minipage}{0.45\textwidth}
\begin{flushright} \large
Ronnie \textsc{Pang}    \\
Eric \textsc{Payette}\\
Alessandro \textsc{Power}    \\
James \textsc{Talarico}    \\
Pragas \textsc{Velauthapillai}    \\
Anhkhoi \textsc{Vu-Nguyen}    \\

\end{flushright}
\end{minipage}\\[2cm]

% If you don't want a supervisor, uncomment the two lines below and remove the section above
%\Large \emph{Author:}\\
%John \textsc{Smith}\\[3cm] % Your name

%----------------------------------------------------------------------------------------
%	DATE SECTION
%----------------------------------------------------------------------------------------

{\large \today}\\[1cm] % Date, change the \today to a set date if you want to be precise

%----------------------------------------------------------------------------------------

%----------------------------------------------------------------------------------------
%	LOGO SECTION
%----------------------------------------------------------------------------------------

\includegraphics[scale=0.25]{logo.png} % Include a department/university logo - this will require the graphicx package
 
%----------------------------------------------------------------------------------------
\vfill % Fill the rest of the page with whitespace

\end{titlepage}

%%%%%%%%%%%%%%%%%%%%%%%%%%%%%%%%%%%%%%%%%%%%%%%%%%%%%%%%%%%%%%%%%%%%%%%%%%%%%%%%%%%%%%%%%%%%%%%%%%%%%%%%%%%%%%%%%%%%
%%%%%%%%%%%%%%%%%%%%%%%%%%%%%%%%%%%%%%%%%%%%%%%%%%%%%%%%%%%%%%%%%%%%%%%%%%%%%%%%%%%%%%%%%%%%%%%%%%%%%%%%%%%%%%%%%%%%
%%%%%%%%%%%%%%%%%%%%%%%%%%%%%%%%%%%%%%%%%%%%%%%%%%%%%%%%%%%%%%%%%%%%%%%%%%%%%%%%%%%%%%%%%%%%%%%%%%%%%%%%%%%%%%%%%%%%
%%%%%%%%%%%%%%%%%%%%%%%%%%%%%%%%%%%%%%%%%%%%%%%%%%%%%%%%%%%%%%%%%%%%%%%%%%%%%%%%%%%%%%%%%%%%%%%%%%%%%%%%%%%%%%%%%%%%

%%%%%%%%%%%%%%%%%%%%%%%%%%%%%%%%%%%%%%%%%%%%%%%%%%%%%%%%%%%%%%%%%%%%%%%%%%%%%%%%%%%%%%%%%%%%%%%%%%%%%%%%%%%%%%%%%%%%
                                            %---------------------------%
                                            %   Project Description     %
                                            %---------------------------%
%%%%%%%%%%%%%%%%%%%%%%%%%%%%%%%%%%%%%%%%%%%%%%%%%%%%%%%%%%%%%%%%%%%%%%%%%%%%%%%%%%%%%%%%%%%%%%%%%%%%%%%%%%%%%%%%%%%%
\vfill
\newpage
\section{Project Description}
Mytinerary is a web application designed to create, adjust and optimize the schedules of Concordia students. The schedule is created by the students who add courses offered by their respective program. Prior to the creation of the schedule, the user must select the according school term. In addition, the program will check whether or not the students have the correct pre or co-requisites prior to adding the class to the schedule. What this program differs from the current Concordia schedule making, is the friendly user interface. One major example is that it displays what the current schedule looks like while the student adds courses. The reason behind this is that the student can add courses while viewing their available schedule times rather than going back and forth viewing what time space is free. The application will also prohibit students from adding courses that is out of their course sequence or do not possess the correct pre/co-requisite.
This application is also usable by teachers alike. Teachers can post their term schedule which includes the courses being taught or their free hours. This is viewed by the students who then join the course.

%%%%%%%%%%%%%%%%%%%%%%%%%%%%%%%%%%%%%%%%%%%%%%%%%%%%%%%%%%%%%%%%%%%%%%%%%%%%%%%%%%%%%%%%%%%%%%%%%%%%%%%%%%%%%%%%%%%%
                                            %---------------------------%
                                            %   Goals and Constraints   %
                                            %---------------------------%
%%%%%%%%%%%%%%%%%%%%%%%%%%%%%%%%%%%%%%%%%%%%%%%%%%%%%%%%%%%%%%%%%%%%%%%%%%%%%%%%%%%%%%%%%%%%%%%%%%%%%%%%%%%%%%%%%%%%
\vfill
\newpage
\section{Goals and Constraints}
The project goals can be divided into two categories. The first consists of
mandatory goals which must be implemented in order to meet the minimum project
requirements. The second category includes goals which, while not absolutely necessary, would
enhance the quality of the project and set Mytinerary apart from other program planners.
These secondary goals were set by asking ourselves what features \emph{we} would look for
in a program planner. Additionally, a survey targeting Concordia software engineering
students was conducted to gauge interest in proposed features, including multilingual
support and integration with Rate-My-Professor. The data from this survey are
presented in Appendix A.

\subsection{Functional Requirements}
The following use cases summarize Mytinerary's functional requirements. Each
use case will deal with two possible \textbf{Actors:} a student, who will use the
system to generate course schedules, and a program director, who can modify
course and section information.

Difficult and importance are scored on a scale of 1 to 10 with 10 denoting extremely high difficulty/importance.\\

\paragraph*{Use case:} Log in\\
\textbf{Actors:} Student, program director\\
\textbf{Goal:} Gain access to website services.\\
\textbf{Preconditions:} User must be a registered user in the system's database.\\
\textbf{Summary:} User logs into the website by providing their username and password.\\
\textbf{Trigger:} User enters their username and password into the respective fields.\\
\textbf{Basic course of events:}
\begin{enumerate}
\item System prompts the user to log in when the user accesses the website.
\item User enters their username and password.
\item System verifies the login information.
\item User is logged in.
\end{enumerate}
\textbf{Exception paths:}
\begin{enumerate}
\item System prompts the user to log in when the user accesses the website.
\item User enters their username and password.
\item System determines that either the username or password were entered incorrectly.
\item User is informed of the failed login attempt and is asked to re-enter their
user credentials.
\end{enumerate}
\textbf{Postconditions:} User is logged in, meaning that they have access to restricted content, including the user's private data. This is accomplished by giving the user anauthentification cookie.\\
\textbf{Difficulty:} 3\\
\textbf{Importance:} 10\\


\paragraph*{Use case:} Log out \\
\textbf{Actors:} Student, program director\\
\textbf{Goal:} Prevent future users of the browser section to have access to user's credentials.\\
\textbf{Preconditions:} User must be logged in. That is, user must have the required authentification cookie.\\
\textbf{Summary:} User logs out of the system, preventing anyone who uses their browser afterwards from having access to the site under the user's name.\\
\textbf{Trigger:} User indicates intention to log off.\\
\textbf{Basic course of action:}
\begin{enumerate}
\item User indicates intention to log off.
\item User if logged off of the system.
\item User is directed to the website's login page.
\end{enumerate}
\textbf{Postconditions:} User is logged off, meaning that they no longer have access to restricted content such as the user's private data. This is accomplished by having the system no longer recognize the previous authentification cookie.\\
\textbf{Difficulty:} 2\\
\textbf{Importance:} 10\\


\paragraph*{Use case:} Return to home page.\\
\textbf{Actors:} Student, program director.\\
\textbf{Goal:} Return to the website's home page.\\
\textbf{Precondition:} User must be logged in. That is, user must have the required authentification cookie.\\
\textbf{Summary:} User indicates their intention to return to the home page in order to access all of the site's facilities.\\
\textbf{Trigger:} User indicates intention to return to the home page.\\
\textbf{Basic course of events:}
\begin{enumerate}
\item User indicates intention to return to home page.
\item User is brought to home page.
\end{enumerate}
\textbf{Difficulty:} 2\\
\textbf{Importance:} 8\\


\paragraph*{Use case:} View academic record\\
\textbf{Actors:} Student\\
\textbf{Goal:} View personal academic record.\\
\textbf{Preconditions:} User must be logged in and in the home page.\\
\textbf{Summary:} User indicates that they wish to view their academic record, at which point the system retrieves and displays their academic record.\\
\textbf{Trigger:} User indicates that they wish to view their academic record.\\
\textbf{Basic course of events:}
\begin{enumerate}
\item User indicates intention to view academic record.
\item Academic record is presented to user.
\end{enumerate}
\textbf{Difficulty:} 2\\
\textbf{Importance:} 7\\


\paragraph*{Use case:} View student's academic record\\
\textbf{Actors:} Program director\\
\textbf{Goal:} View the academic record of a selected student.\\
\textbf{Precondition:} User must be a program director. Student must exist in the system's database.\\
\textbf{Summary:} Program director requests access to academic record of selected student, which is then displayed.\\
\textbf{Trigger:} Program director indicates intention to view a student's academic record.\\
\textbf{Basic course of events:}
\begin{enumerate}
\item User indicates intention to view academic record.
\item System prompts user to enter either the name of the student or the student's Concordia ID number.
\item System verifies that user is a program director.
\item System verifies that specified student exists.
\item System retrieves student's academic record and displays it to the user.
\end{enumerate}
\textbf{Exception paths:}\\
\textbf{Student does not exist:}
\begin{enumerate}
\item User indicates intention to view academic record.
\item System prompts user to enter either the name of the student or the student's Concordia ID number.
\item System verifies that user is a program director.
\item System determines that specified student does not exist.
\item User is informed that specified student does not exist, and is prompted to re-enter the student's identifying information
\end{enumerate}.
\textbf{User is not a program director:}
\begin{enumerate}
\item User indicates intention to view academic record.
\item System prompts user to enter either the name of the student or the student's Concordia ID number.
\item System determines that the user is not a program director. This shouldn't happen, as non-program directors will not be given the opportunity to even attempt to view a student's academic record other than their own; however, for extra security the system will perform this check.
\item User presented with an error message and is immediately logged off.
\end{enumerate}
\textbf{Difficulty:} 3\\
\textbf{Importance:} 6\\


\paragraph*{Use case:} View course listing\\
\textbf{Actors:} Student, program director\\
\textbf{Goal:} View a list of available courses.\\
\textbf{Summary:} A list of all available courses, grouped by program and sorted in numerical order, is presented to the user.\\
\textbf{Trigger:} User indicates intention to see course list.\\
\textbf{Basic course of events:}
\begin{enumerate}
\item User indicated intention to see course list.
\item System retrieves list of courses and displays them to user in appropriate order.
\end{enumerate}
\textbf{Difficulty:} 2\\
\textbf{Importance:} 8\\


\paragraph*{Use case:} View academic progress\\
\textbf{Actors:} Student\\
\textbf{Goal:} View current progress towards completing degree in Software Engineering.\\
\textbf{Precondition:} User is logged in as a user. That is, user must have the required authentification cookie.\\
\textbf{Included Use case:} View course listing.\\
\textbf{Summary:} When accessing list of courses, system will present a summary of user's progress towards their Software Engineering degree.\\
\textbf{Trigger:} User indicates intention to see course list.\\
\textbf{Basic course of events:}
\begin{enumerate}
\item User indicates intention to see course list.
\item System retrieves list of courses (see \paragraph*{Use case:} View course listing).
\item System accesses student's academic record to determine courses that have been
completed.
\item System compares completed courses versus required courses.
\item System displays course list to user along with information highlighting their
academic progress.
\end{enumerate}
\textbf{Difficulty:} 4\\
\textbf{Importance:} 6\\


\paragraph*{Use case:} View course details.\\
\textbf{Actors:} Student, program director.\\
\textbf{Goal:} View details of specified course.\\
\textbf{Precondition:} Course exists in the system's database.\\
\textbf{Summary:} Users enters a course number to view details such as location, number of credits, prerequisites, etc.\\
\textbf{Trigger:} User enters course number into appropriate field.\\
\textbf{Basic course of events:}
\begin{enumerate}
\item User indicates intention to view course details.
\item User enters course number into appropriate field.
\item System retrieves course information and displays it to user.
\end{enumerate}
\textbf{Exception paths:}\\
\textbf{Course does not exist:}
\begin{enumerate}
\item User indicates intention to view course details.
\item User enters course number into appropriate field.
\item System does not find any course matching that course number.
\item User is informed that the specified course does not exist, and is re-prompted
to enter a course number.
\end{enumerate}
\textbf{Difficulty:} 3\\
\textbf{Importance:} 8\\


\paragraph*{Use case:} Modify course details.\\
\textbf{Actors:} Program director.\\
\textbf{Goal:} Modify the details of a course.\\
\textbf{Extended Use case:} View course details\\
\textbf{Precondition:} User is a program director. Course exists in the system's database.\\
\textbf{Summary:} User specifies a course to modify. Properties of the course that can be modified include location, time, professor, class capacity, etc.\\
\textbf{Trigger:} User enters course number into appropriate field.\\
\textbf{Basic course of events:}
\begin{enumerate}
\item User indicates intention to modify course details.
\item User enters course number into appropriate field.
\item System retrieves course information and displays it to user.
\item System displays fields where user can edit course information.
\item User enters new course information.
\item System checks that user is a program director.
\item System modifies course information based on user input.
\end{enumerate}
\textbf{Exception paths:}\\
\textbf{Course does not exist:}
\begin{enumerate}
\item User indicates intention to modify course details.
\item User enters course number into appropriate field.
\item System does not find any course matching that course number.
\item User is informed that the specified course does not exist, and is re-prompted
to enter a course number.
\end{enumerate}
\textbf{User is not a program director:}
\begin{enumerate}
\item User indicates intention to modify course details.
\item User enters course number into appropriate field.
\item System retrieves course information and displays it to user.
\item System displays fields where user can edit course information.
\item User enters new course information.
\item System confirms that user is not a program director. An error message is displayed
and the user is logged off the system. This should not happen in practice, as the
user should not be given the opportunity to even attempt to modify course information.
\end{enumerate}
\textbf{Difficulty:} 3\\
\textbf{Importance:} 7\\


\paragraph*{Use case:} View news feed\\
\textbf{Actors:} Student, program director\\
\textbf{Goal:} Present users with a news feed.\\
\textbf{Precondition:} User is logged in and on the home page.\\
\textbf{Summary:} User is presented with headlines of news around the world.\\
\textbf{Trigger:} User is on the home page.\\
\textbf{Basic course of events:}
\begin{enumerate}
\item User is on the home page and is presented with headlines of current news.
\end{enumerate}
\textbf{Difficulty:} 5\\
\textbf{Importance:} 2\\


\paragraph*{Use case:} Set schedule preferences\\
\textbf{Actors:} Student\\
\textbf{Goal:} User sets their preferences for the type of schedule they want.\\
\textbf{Precondition:} User is logged in.\\
\textbf{Summary:} User sets their schedule preferences, which may include their available times, days they cannot go to school on, number of credits they wish to take, etc.\\
\textbf{Trigger:} User requests to edit schedule preferences.\\
\textbf{Basic course of events:}
\begin{enumerate}
\item User requests to edit schedule preferences.
\item System presents user with an interface to set preferences.
\item User uses this interface to set their preferences, which the system then saves.
\end{enumerate}
\textbf{Postconditions:} User preferences are saved in the database.\\
\textbf{Difficulty:} 3\\
\textbf{Importance:} 9\\


\paragraph*{Use case:} Select course for schedule\\
\textbf{Actors:} Student\\
\textbf{Goal:} User selects a course they wish to be added to their next schedule.\\
\textbf{Precondition:} User is logged in and they are eligible to take the specified course.\\
\textbf{Included use cases:} View course listing OR view course details.\\
\textbf{Summary:} User adds a course they wish to be included in their next schedule.\\
\textbf{Trigger:}\\
\textbf{Basic course of events:}
\begin{enumerate}
\item User selects a course to be added, either by searching specifically for the course
(see use case View course details) or by selecting the course in the list of available courses
(see use case View course listing).
\end{enumerate}
\textbf{Difficulty:} 4\\
\textbf{Importance:} 9\\

\subsubsection{Domain Model}
Figure \ref{fig:domain_model} illustrates a domain model for our system. The principal domain objects are the student, the course section, and the scheduler itself. Students, who are users, have preferences and an academic record and interact with the scheduler. The scheduler in turn consults the student's academic record and preferences, the requirements of the program the student is enrolled in, and the available course sections offered in each semester to generate a list of tentative schedules for the student to select. Course sections consist possibly of a tutorial section and a lab and are in turn instances of the more general course object. Finally there are program directors, who are users of the system with the privilege to view and modify student academic records and manage course details.


%%%%%%%%%%%%%%%%%%%%%%%%%%%%%%%%%%%%%%%%%%%%%%%%%%%%%%%%%%%%%%%%%%%%%%%%%%%%%%%%%%%%%%%%%%%%%%%%%%%%%%%%%%%%%%%%%%%%
                                            %---------------------------%
                                            %   Resource Requirements   %
                                            %---------------------------%
%%%%%%%%%%%%%%%%%%%%%%%%%%%%%%%%%%%%%%%%%%%%%%%%%%%%%%%%%%%%%%%%%%%%%%%%%%%%%%%%%%%%%%%%%%%%%%%%%%%%%%%%%%%%%%%%%%%%
\vfill
\newpage
\section{Resource Requirements}

\subsection{Resource Evaluation}
\begin{center}
\begin{tabular}{ p{2.7cm} | p{9cm} }
\hline
\textbf{Member Name}	&	\textit{\textbf{Laurendy Lam}}	\\ \hline \hline
\textbf{Role}		&	Programing Full-Stack, Programming Leader, Co-Team Leader	\\ \hline
\textbf{Strengths}	&	-- Web Design	\\
			        &	-- User interface	\\
			        &	-- Algorithm	\\
					&	-- Team Work / Organization	\\
					&	-- Database Design	\\
					&	-- OOP	\\ \hline
\textbf{Knowledge}	&	Java, C, C++, PHP, HTML5, CSS3, Javascript, SQL, Python, Ruby, Prolog, Lisp	\\ \hline
\textbf{Experience}	&	-- Personal website developer	\\
					&	-- Design/Developed a restaurant website 	\\ \hline
\textbf{Availability}	&	10-12 hours /week	\\ \hline
\end{tabular}
\end{center}
%
\vspace{3mm}
%
\begin{center}
\begin{tabular}{ p{2.7cm} | p{9cm} }
\hline
\textbf{Member Name}	&	\textit{\textbf{Alessandro Power}}	\\ \hline \hline
\textbf{Role}		&	Documentation Leader , Team Leader	\\ \hline
\textbf{Strengths}	&	-- Communication	\\
					&	-- Team Work	\\
					&	-- Object-oriented Programming	\\ \hline
\textbf{Knowledge}	&	C++, Java, Python	\\ \hline
\textbf{Experience}	&	Programming in Java for various class assignments.	\\ \hline
\textbf{Availability}	&	7 hours/week	\\ \hline
\end{tabular}
\end{center}
%
\vspace{3mm}
%
\begin{center}
\begin{tabular}{ p{2.7cm} | p{9cm} }
\hline
\textbf{Member Name}	&	\textit{\textbf{Anhkhoi Vu-Nguyen}}	\\ \hline \hline
\textbf{Role}		&	Programming Back-End	\\ \hline
\textbf{Strengths}	&	-- Object-oriented Programming	\\
					&	-- Team Work	\\
					&	-- Communication	\\ \hline
\textbf{Knowledge}	&	Java, C++, HTML5, Prolog, PHP, Lisp, CSS3, Assembly , AspectJ	\\ \hline
\textbf{Experience}	&	-- Programming in Java for various class assignments 	\\ \hline
\textbf{Availability}	&	6 hours/week	\\ \hline
\end{tabular}
\end{center}
%
\vspace{3mm}
%
\begin{center}
\begin{tabular}{ p{2.7cm} | p{9cm} }
\hline
\textbf{Member Name}	&	\textit{\textbf{Jacqueline Luo}}	\\ \hline \hline
\textbf{Role}		&	Testing	\\ \hline
\textbf{Strengths}	&	-- OOP with Java	\\
					&	-- App development for Android devices	\\ \hline
\textbf{Knowledge}	&	Java, C\#, HTML , CSS3, PHP, JavaScript	\\ \hline
\textbf{Experience}	&	Developed an app on Google Play Store	\\ \hline
\textbf{Availability}	&	6 hours/week	\\ \hline
\end{tabular}
\end{center}
%
\vspace{3mm}
%
\begin{center}
\begin{tabular}{ p{2.7cm} | p{9cm} }
\hline
\textbf{Member Name}	&	\textit{\textbf{James Talarico}}	\\ \hline \hline
\textbf{Role}		&	Documentation	\\ \hline
\textbf{Strengths}	&	-- Object-Oriented Design	\\
					&	-- Communication and teamwork	\\
					&	-- Document Writing	\\ \hline
\textbf{Knowledge}	&	Java, C, C++ , Python , OCaml, HTML5, CSS3, MATLAB, Bash	\\ \hline
\textbf{Experience}	&	-- Developed a cluster search algorithm to parse Fermi LAT data for uncatalogued VHE sources.	\\
					&	-- Configured McPHAC to run on Calcul Quebec's super computers.	\\ \hline
\textbf{Availability}	&	6 hours/week	\\ \hline
\end{tabular}
\end{center}
%
\vspace{3mm}
%
\begin{center}
\begin{tabular}{ p{2.7cm} | p{9cm} }
\hline
\textbf{Member Name}	&	\textit{\textbf{Kenny Nguyen}}	\\ \hline \hline
\textbf{Role}		&	Documentation	\\ \hline
\textbf{Strengths}	&	-- Object-oriented Programming	\\
					&	-- Team Work	\\
					&	-- Communication	\\ \hline
\textbf{Knowledge}	&	Java, PHP, HTML5, CSS3, MySQL, JavaScript 	\\ \hline
\textbf{Experience}	&	-- Programmer at Industry Canada	\\ \hline
\textbf{Availability}	&	6 hours/week	\\ \hline
\end{tabular}
\end{center}
%
\vspace{3mm}
%
\begin{center}
\begin{tabular}{ p{2.7cm} | p{9cm} }
\hline
\textbf{Member Name}	&	\textit{\textbf{Michael Mescheder}}	\\ \hline \hline
\textbf{Role}		&	Programming Full-Stack	\\ \hline
\textbf{Strengths}	&	-- Client side	\\
					&	-- Object-Oriented Programming	\\
					&	-- Teamwork	\\ \hline
\textbf{Knowledge}	&	Java, PHP, HTML5, CSS3, JavaScript	\\ \hline
\textbf{Experience}	&	-- Programming in Java for various class assignments.	\\ \hline
\textbf{Availability}	&	6 hours/week	\\ \hline
\end{tabular}
\end{center}
%
\vspace{3mm}
%
\begin{center}
\begin{tabular}{ p{2.7cm} | p{9cm} }
\hline
\textbf{Member Name}	&	\textit{\textbf{Piratheeban Annamalai}}	\\ \hline \hline
\textbf{Role}		&	Programming	\\ \hline
\textbf{Strengths}	&	-- Server side	\\
					&	-- Object-Oriented Programming	\\
					&	-- Organization	\\ \hline
\textbf{Knowledge}	&	Java, PHP, HTML5, CSS3, JavaScript	\\ \hline
\textbf{Experience}	&	-- Various programming assignments for class assignments involving Java, and Php.	\\ \hline
\textbf{Availability}	&	6 hours/week	\\ \hline
\end{tabular}
\end{center}
%
\vspace{3mm}
%
\begin{center}
\begin{tabular}{ p{2.7cm} | p{9cm} }
\hline
\textbf{Member Name}	&	\textit{\textbf{Pragas Velauthapillai}}	\\ \hline \hline
\textbf{Role}		&	Testing	\\ \hline
\textbf{Strengths}	&	-- OOP	\\
					&	-- Team Work	\\ \hline
\textbf{Knowledge}	&	Java, PHP, HTML, CSS, Javascript	\\ \hline
\textbf{Experience}	&	-Programming in Java for various class assignments.	\\ \hline
\textbf{Availability}	&	6 hours/week	\\ \hline
\end{tabular}
\end{center}
%
\vspace{3mm}
%
\begin{center}
\begin{tabular}{ p{2.7cm} | p{9cm} }
\hline
\textbf{Member Name}	&	\textit{\textbf{Ronnie Pang}}	\\ \hline \hline
\textbf{Role}		&	Programming Full Stack	\\ \hline
\textbf{Strengths}	&	-- OOP	\\
					&	-- Team Work	\\ \hline
\textbf{Knowledge}	&	Java, PHP, HTML, CSS, Javascript	\\ \hline
\textbf{Experience}	&	-- Programming in Java for various class assignments.	\\ \hline
\textbf{Availability}	&	6 hours/week	\\ \hline
\end{tabular}
\end{center}
%
\vspace{3mm}
%
\begin{center}
\begin{tabular}{ p{2.7cm} | p{9cm} }
\hline
\textbf{Member Name}	&	\textit{\textbf{Eric Payette}}	\\ \hline \hline
\textbf{Role}		&	Programming Full-Stack	\\ \hline
\textbf{Strengths}	&	-- Object-oriented programming	\\
					&	-- Functional programming	\\
					&	-- Logic programing	\\
					&	-- Team Work	\\ \hline
\textbf{Knowledge}	&	Java, PHP , HTML , CSS3, MySQL , JavaScript, Lisp, Prolog, AspectJ	\\ \hline
\textbf{Experience}	&	-- Programming for various class assignment.	\\ \hline
\textbf{Availability}	&	6 hours/week	\\ \hline
\end{tabular}
\end{center}
%
\vspace{3mm}
%
\begin{center}
\begin{tabular}{ p{2.7cm} | p{9cm} }
\hline
\textbf{Member Name}	&	\textit{\textbf{Andy Nguyen}}	\\ \hline \hline
\textbf{Role}		&	Documentation	\\ \hline
\textbf{Strengths}	&	-- Object-oriented Programming	\\
					&	-- Team Work	\\ \hline
\textbf{Knowledge}	&	Java, HTML5 , CSS3, JavaScript , PHP	\\ \hline
\textbf{Experience}	&	-- Programming in Java for class assignments	\\
					&	-- Developed a Real-Estate website for a class project	\\ \hline
\textbf{Availability}	&	5 hours/week	\\ \hline
\end{tabular}
\end{center}
%
\vspace{3mm}
%
\subsection{Technical Resources}

\subsubsection{Server}

The web server used to host the developing project is the PHP built in web server provided by PHP which come with WAMP. It comes bundle of different other tools such as PHPmyAdmin, Apache and SQL buddy for local hosting.
\subsubsection{Database}

 The hosting service for the database on a remote server is Heroku. To manage the database we are using dbForge studio for MySQL or PHPMyAdmin version 4.5.3 used with WAMP to run the app to access the database remotely.

\subsubsection{Hardware}

 The average computer to host the working project is an Intel i5 or similar. 2 GB of ram and uses around 30 MB of storage space. 

\subsubsection{System}

The operating system used to operate the program as very portable as long as it has PHP, Apache and a MySQL server install installed.  Version and course control is handled by a private git repository. 

\subsubsection{Programming Languages}
The language used to develop this project is PHP, JavaScript, and CSS. The framework used to develop the project is CodeIgniter v3.03 along with other user interface frameworks such as JQuery v3.03, Bootstrap  4, Normalize v3.0.2  and Google Fonts. The editors to develop the project are: notepad++ or PHPStorm by JetBrains. Refer to section \ref{techInUse} for a more extensive list of the technologies in use.

\subsubsection{Communication}

We are using slack to communicate, Google drive to share files and documents. and GitHub to upload finished documents. bug reporting on git hub. For the documentation we are using LaTeX. We use this to produce a nice compilation of all the documents collected throughout the project.  

\vfill
\newpage
\section{Scoping}

%%%%%%%%%%%%%%%%%%%%%%%%%%%%%%%%%%%%%%%%%%%%%%%%%%%%%%%%%%%%%%%%%%%%%%%%%%%%%%%%%%%%%%%%%%%%%%%%%%%%%%%%%%%%%%%%%%%%
                                            %---------------------------%       
                                            %     Solution Sketch       %
                                            %---------------------------%
%%%%%%%%%%%%%%%%%%%%%%%%%%%%%%%%%%%%%%%%%%%%%%%%%%%%%%%%%%%%%%%%%%%%%%%%%%%%%%%%%%%%%%%%%%%%%%%%%%%%%%%%%%%%%%%%%%%%
\vfill
\newpage
\section{Solution Sketch}

\subsection{Architecture}
%
\begin{figure}[ht!]
\centering
\includegraphics[width=0.7\columnwidth]{MVCDiagram.png}
\caption{Diagram of the Model View Controller which Mytinerary is based on. \label{MVC}}
\end{figure}
%
Myitinerary is based off of the Model-View-Controller (MVC) architecture. We are using MVC because it encourages a clear separation between the data and the methods that manipulate the data. The gives our program the advantage of allowing a front-end and back-end developer to work on the project at the same time, which out having to worry about their respective code interfering with one another.

As the name suggests, the Model-View-Controller pattern can be broken down into three layers: Model, View and Controller. These layers are what create an abstraction between components of the system as well as a layer of abstraction between the system and the user. A diagram of the interactions of the MVC can be seen in Figure \ref{MVC}. The three layers are described below.
%
\paragraph*{Model}~\\
The Model is where the data used in the program is permanently stored. The model acts as a sort of bridge between the view and controller component. It updates the View on pertinent changes that have been made, and allows the Controller to manipulate it's data content. 
\paragraph*{View}~\\
The View is where the information requested from the Model layer can be viewed. In our system, this would correspond to Mytinerary's HTML page. The View does not receive any information from the Controller whatsoever. All data received by the View is given upon requesting it from the Model. This happens when a method from the Controller changes data in the Model. The Model then informs the View of these changes.
%
\paragraph*{Controller}~\\
The job of the Controller is to take in any information that is passed by the user, and update the Model accordingly. The Controller does not interact with the View directly, but instead changes the data held by the Model. It is important to note that changes made by the Controller are initiated by user interactions.
%
\subsection{Technologies in Use} \label{techInUse}
%
%
\subsubsection{Programming Languages} \label{languages}
\paragraph*{PHP}~\\
We decided to use PHP as the code base for Mytinerary. It is a quick and easy server side scripting language. PHP has a large community, which meant that most problems faced by developers would have preexisting solutions. Another important advantage is that PHP is part of the development platform WAMP, as described in section \ref{webanddata}, which made installation easy. We decided to use the most recent version, PHP 7.02.
%
\paragraph*{JavaScript}~\\
JavaScript is a high-level, dynamic and untyped programming language. Since JavaScript is commonly used for web development, it is supported by most browsers, including: Internet Explorer, Chrome, Firefox and Safari. In order to transfer data, we will be using JSON (JavaScript Object Notation) in place of the more common XML. This choice was motivated by the fact that JSON is more lightweight, making network transmissions and read/writes faster. JSON is also far more human-readable compared to XML. We used the stable release version ECMAScript 6.
%
\paragraph*{HTML}~\\
HTML (Hyper Text Markup Language) is a markup language used for creating web pages. It is by far the most commonly used markup language, and therefore has the most cross-platform support. We used the most recent version HTML5, which is quickly rising in popularity. 
%
\paragraph*{CSS}~\\
CSS (Cascading Style Sheets) is a style sheet language used for creating a layout and styling a web page. Like JavaScript and HTML, CSS is one of the more common languages used in web development. CSS is used primarily to allow the separation of a web page's content from it's presentation; this makes code cleaner and more reusable. We are using CSS3, the most recent version.
%
\paragraph*{SQL}~\\
SQL (Structured Query Language) is a special-purpose programming language used to access and manipulate databases. We will be using it to manage our MySQL database (described in section \ref{webanddata}, under MySQL). We chose SQL since it makes retrieving large amounts of records from databases quick and efficient it is the standard language for this task. We are using the most recent version, SQL:2011.
%
%
%
\subsubsection{Libraries and Frameworks}\label{framworks}
\paragraph*{Bootstrap}~\\
Bootstrap is a HTML, CSS, and JavaScript framework used for developing web pages. We chose Bootstrap as it is easy to use and has a large community with extensive documentation. Furthermore, Bootstrap supports the most popular browsers and as well as fixes some compatibility issues with CSS. We are using the most recent stable release, Bootstrap 3.3.6.
%
\paragraph*{CodeIgniter}~\\
CodeIgniter is an open source PHP framework used in web development. It is loosely based on the Model-View-Controller pattern. We chose CodeIgniter as it is easy to use compared to other PHP frameworks and was the framework that our developers were most familiar with. We are using the most recent stable release, 3.0.4.
%
\paragraph*{AJAX}~\\
AJAX (Asynchronous JavaScript and XML) is a set of client-side web development techniques used to create asynchronous Web applications. We chose AJAX over AJAJ (which uses JSON) for several reasons, the first being our programing team was more familiar with it. AJAJ also has a lack of documentation and is not able to perform two-way data transmission, so AJAX was the clear choice.
%
\paragraph*{jQuery}~\\
jQuery is a JavaScript library which is designed to simplify client-side HTML. We chose jQuery as it is easy to use, especially compared to other JavaScript libraries. It is an extensive library that effortlessly supports AJAJ. We are using jQuery 1.12.0.
%
\paragraph*{Normalize.css}~\\
Normalize.css is used to make built-in browser styling consistent across browsers. It also has a very extensive documentation and excellent support, which was very helpful for some of the members which were not familiar with the tool. We are using version 3.0.3.
%
\paragraph*{Moment.js}~\\
Moment.js is a simple date and time library that enables us manipulate time and dates very easily. It comes with extensive documentation and a variety of features, unlike JavaScript's native date library.
%
%
%
\subsubsection{Web server and Databases} \label{webanddata}
\paragraph*{Apache}~\\
Apache is an open-source web server software. We chose it because it is currently the most popular web server software and is the one we were most familiar with. We will be using the stable release, 2.4.18, which is the most recent version.
%
\paragraph*{Heroku}~\\
Heroku is a Platform-as-a-Service (PaaS), which allows it's users to host a database on a remote server. We chose Heroku because other web hosting services were not working properly with Mytinerary. Heroku was user-friendly and had extensive documentation making it easy to learn.
\todo[inline] {Add this!}
%
\paragraph*{MySQL}~\\
MySQL is a Database Management system which uses relational databases. We chose MySQL as it is easy to use, very popular and has excellent support. We also chose it because our programming team was very familiar with it. We will be using the most recent stable release, 5.7.10.
%
\paragraph*{PHPMyAdmin}~\\
PHPMyAdmin is a web based tool that works to help the user handle the administration of MySQL. We chose it because of it's simplicity and our familiarity with software. We are using the most recent stable release version, 4.5.4.1.
%
\paragraph*{dbForge Studio}~\\
dbForge Studio is a windows tool that is made for management and development of an SQL server. We chose this tool because are our familiarity with it. We are using the most recent version, 5.1.178.
%
\paragraph*{WAMPServer}~\\
WAMPServer is a windows development environment which uses Apache, MySQL, and PHP. Since we had already decided on using PHP, Apache and MySQL for our project, using this tool to simplify the setup of our server was the obvious choice. We are using the most recent stable release version, 4.5.4.1.
%
\subsubsection{IDEs}
\paragraph*{Notepad++}~\\
Notepad++ is a simple, lightweight text editor that has support for a large range of languages. We chose it because it is a very popular cross-platform text editor and our programming team was very familiar with it. We are using Notepad++ v6.8.8.
%
\paragraph*{PHPStorm}~\\
PHPStorm is an IDE which handles PHP, HTML and JavaScript. We chose it because it is cross-platform and was the IDE our programming team was the most familiar with it. We are using version 10.0.3.
%
%
%
\subsubsection{Documentation and Collaboration Software}
\paragraph*{Slack}~\\
Slack is an online messaging application that can be used on mobile or PC. We chose to use Slack since it made communication and collaboration more convenient and organized.
%
\paragraph*{GitHub}~\\
GitHub is an online Git repository service. It is currently the most popular Git based system, and it virtually essential for version control in large programming project. GitHub is an excellent tool for organizing and collaborating, and is an industry standard.
%
\paragraph*{ShareLaTeX}~\\
ShareLaTeX is an online LaTeX editor and compiler. We chose to use LaTeX because it's capability of producing high typographical quality documentation. We chose ShareLaTeX since it allowed the documentation team to collaborate easily and it they were already quite familiar with it. 
%
\paragraph*{Gliffy}~\\
Gliffy is an online web diagram editor with was used to make all UMLs and Domain Models in during the project. We chose it because it is easy to use and can produce high-quality diagrams.
%
%%%%%%%%%%%%%%%%%%%%%%%%%%%%%%%%%%%%%%%%%%%%%%%%%%%%%%%%%%%%%%%%%%%%%%%%%%%%%%%%%%%%%%%%%%%%%%%%%%%%%%%%%%%%%%%%%%%%
                                             %---------------------------%       
                                             %           Plan            %
                                             %---------------------------%
%%%%%%%%%%%%%%%%%%%%%%%%%%%%%%%%%%%%%%%%%%%%%%%%%%%%%%%%%%%%%%%%%%%%%%%%%%%%%%%%%%%%%%%%%%%%%%%%%%%%%%%%%%%%%%%%%%%%
\vfill
\newpage
\section{Plan}


%%%%%%%%%%%%%%%%%%%%%%%%%%%%%%%%%%%%%%%%%%%%%%%%%%%%%%%%%%%%%%%%%%%%%%%%%%%%%%%%%%%%%%%%%%%%%%%%%%%%%%%%%%%%%%%%%%%%
                                             %---------------------------%       
                                             %       Prototyping         %
                                             %---------------------------%
%%%%%%%%%%%%%%%%%%%%%%%%%%%%%%%%%%%%%%%%%%%%%%%%%%%%%%%%%%%%%%%%%%%%%%%%%%%%%%%%%%%%%%%%%%%%%%%%%%%%%%%%%%%%%%%%%%%%
\vfill
\newpage

\section{Prototyping}

Upon entering the website’s URL the user is presented with a login page. The login page features a brief description of the current situation. For instance, any course issues or service downtime would be displayed here. The users also have the ability to view courses in any given semester without needing to log in. The login system security is handled by hashing the entered username and password using the Blowfish algorithm and sent to the database for server side validation. 
Once logged in, the users are sent to the default page with various options presented to them.  Additionally, there is also a section containing relevant news that will updated on a regular interval. This section is currently created by using JSON and the New York Times API (may be modified later on). 

The first selectable option is the student profile. This page shows a record of the student. It indicates which courses have been completed and the GPA. Also, this page is used to determine which prerequisites and co-requisites have been completed before a student can register for a class that has those specific prerequisites and co-requisites.

The second selectable option leads to the registration page. Here, students can search for courses in a selected semester in order to generate schedules. There is also an option to auto-generate schedules that is currently in development. Students can apply time preferences to generate their ideal schedule.  Up to five different schedules can be displayed on the same page using client side JavaScript and jQuery to dynamically generate the schedules as the user selects or edits information. The students can then select which schedule they would prefer to have by comparing all of them. Students can add courses to the courses tab. However, in the event that they lack the necessary prerequisites or co-requisites, they will not be able to add the course onto the registered tab. This validation is handled by dynamically checking the students registered courses within the database. The courses shown on the right-hand side represent what the students are involved with. If the course name is within a green rectangle, then the student will be currently registered for that class. If the course name is within a yellow rectangle, then the student will be on a waiting list for that class. Finally, if the course name is within a red rectangle, then it indicates that the student has removed that class from the current semester.

At any time, users can click on the Mytinerary logo on the top left corner to be redirected to the home page. At the home page students can also view available courses as well as their current schedule. Mytinerary is cross-platform compatible in addition to featuring a responsive layout and automatic scaling. This is achieved using HTML5 and Bootstrap as the front-end framework. Any APIs to be used will also avoid being platform specific to ensure usability across all devices.

%%%%%%%%%%%%%%%%%%%%%%%%%%%%%%%%%%%%%%%%%%%%%%%%%%%%%%%%%%%%%%%%%%%%%%%%%%%%%%%%%%%%%%%%%%%%%%%%%%%%%%%%%%%%%%%%%%%%
%%%%%%%%%%%%%%%%%%%%%%%%%%%%%%%%%%%%%%%%%%%%%%%%%%%%%%%%%%%%%%%%%%%%%%%%%%%%%%%%%%%%%%%%%%%%%%%%%%%%%%%%%%%%%%%%%%%%
%%%%%%%%%%%%%%%%%%%%%%%%%%%%%%%%%%%%%%%%%%%%%%%%%%%%%%%%%%%%%%%%%%%%%%%%%%%%%%%%%%%%%%%%%%%%%%%%%%%%%%%%%%%%%%%%%%%%
%%%%%%%%%%%%%%%%%%%%%%%%%%%%%%%%%%%%%%%%%%%%%%%%%%%%%%%%%%%%%%%%%%%%%%%%%%%%%%%%%%%%%%%%%%%%%%%%%%%%%%%%%%%%%%%%%%%%

\vfill
\newpage
\appendix
\section{Source Code}
% The important Source Code
% the ones commented out give errors for some reason
\lstinputlisting[language=php]{webroot/application/controllers/Courses.php}
\lstinputlisting[language=php]{webroot/application/controllers/Home.php}
\lstinputlisting[language=html]{webroot/application/controllers/index.html}
\lstinputlisting[language=php]{webroot/application/controllers/Login.php}
\lstinputlisting[language=php]{webroot/application/controllers/Students.php}
\lstinputlisting[language=php]{webroot/application/models/Course.php}
\lstinputlisting[language=html]{webroot/application/models/index.html}
\lstinputlisting[language=php]{webroot/application/models/Scheduler.php}
\lstinputlisting[language=php]{webroot/application/models/Section.php}
\lstinputlisting[language=php]{webroot/application/models/Semester.php}
\lstinputlisting[language=php]{webroot/application/models/Student.php}
\lstinputlisting[language=php]{webroot/application/models/User.php}
\lstinputlisting[language=html]{webroot/application/views/index.html}
\lstinputlisting[language=php]{webroot/application/views/course/result.php}
\lstinputlisting[language=php]{webroot/application/views/course/search.php}
%\lstinputlisting[language=php]{webroot/application/views/layouts/footer.php}
%\lstinputlisting[language=php]{webroot/application/views/layouts/header.php}
\lstinputlisting[language=php]{webroot/application/views/login/index.php}
%\lstinputlisting[language=php]{webroot/application/views/student/home.php}
\lstinputlisting[language=php]{webroot/application/views/student/profile.php}
\lstinputlisting[language=php]{webroot/application/views/student/scheduler.php}

\end{document}
