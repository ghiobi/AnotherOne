Database

4.1

Within the database, it's very important to note that the active user
using the app will have a corresponding ID number. This is will be used
during the user's process throughout their activity time. Thus any
action will correspond to that user ID. When accessing the database, the
user is detected to be either admin or students. The database however,
is commonly used when the user is a student, thus allowing them to
view/change their schedule from the database.

Going further into the student objects, it branches out into 2 separate
blocks: registered and program. Registered block contains the list of
the courses the student is currently enrolled in. Furthermore, the
sections block is used to represent the enrolled course's sections which
include: lecture, tutorial and labs. The reason sections, tutorials and
laboratories blocks are linked to the registered block is to achieve a
quicker access to the saved course, as courses within the registered
block will have an existing section ID which includes tutorial and labs
ID if possible. This method is preferred over accessing the section
block first followed by accessing the tutorial/labs sections when
accessing the database.

The other path: program will allow the student to access the correct
program sequence according to the corresponding program. This option is
selected when the student wants to add a new course to their schedule.
Following the program block is the programsequence block which contains
the list of course within the program's sequence. Once in the course
block, it branches out into a few other blocks: subject,
coursecorequisite and courseprequisite blocks. The subject block returns
the code of the course selected from the, ie: SOEN341. The
courseprequisite and coursecorequisite blocks contain any possible
pre/corequisite of that selected course.

Furthermore, the courses block is linked to the sections block , in
order to finally insert the course into the schedule with the same ID as
the original user.

4.2

\begin{itemize}
\item
  \textbf{Users} : Contains a list of every users of this application.
\item
  \textbf{Admins}: Contains list of every ``admin''-type in the users
  table
\item
  \textbf{Students}: Contains list of every ``student''-type in the
  users table
\item
  \textbf{Registered}: contains list of registered courses for a
  selected student
\item
  \textbf{Program}: contains list of every program in Concordia
\item
  \textbf{Programsequence}: list of courses for a selected program.
\item
  \textbf{Courses}: Contains details of a selected general course.
  Details contain: code, co-requisite, pre-requisite and section. This
  does not contain the availability of the class for a semester.
\item
  \textbf{Subject}: Similar to program table, however contains the code
  rather than the name.
\item
  \textbf{Courseprequisite}: contains list of courses that are
  pre-requisite of a selected course.
\item
  \textbf{Coursecorequisite}: contains list of courses that are
  co-requisite of a selected course.
\item
  \textbf{Sections}: contains a list of every lecture sections of
  selected course.
\item
  \textbf{Semesters}: contains details of start and end date of a
  selected course in the selected semester
\item
  \textbf{Lectures}: contains details about a selected lecture section
  (room \#, start time, end time)
\item
  \textbf{Tutorials}: contains a list or every possible tutorial
  sections of the selected lecture section (if exists) and their
  details.
\item
  \textbf{Laboratories}: Contains a list of every possible laboratory
  section of the selected lecture section (if exists) and their details
\end{itemize}
